\section{The paired t-test}

The paired t-test is used to check the difference of means of two conditional samples. This test is usually used to compare "before treatment" and "after treatment".
The tested hypotheses are (where $\delta=\mu_{1}-\mu{2}$)\cite{dodge2008}:
\begin{itemize}
	\item $ H_{0}: \delta=0 $
	no difference between uncuffed and cuffed arm
	\item $ H_{1}: \delta\neq0 $
	a difference between uncuffed and cuffed arm
\end{itemize}
In this case, there is a relation between the microcirculatory and the macrocirculatory system shown, if the null hypothesis is rejected.

It is requisite that the sample size of the of both samples is identical.

Following some useful formulas\cite{dodge2008}:
\begin{itemize}
	\item difference within the subjects, with $ i=1,2,...,n $
	\begin{flalign}
		d_{i}=x_{i2}-x_{i1}
	\end{flalign}
	\item mean of the difference, with $ i=1,2,...,n $
	\begin{flalign}
		\bar{d}=\frac{1}{n}\Sigma d_{i}
	\end{flalign}
	\item standard deviation
	\begin{flalign}
		s_{d}=\sqrt{\frac{\Sigma (d_{i}-\bar{d})^2}{n-1}}
	\end{flalign}
	\item test variable
	\begin{flalign}
		T=\frac{\bar{d}}{\frac{1}{\sqrt{n}}s_{d}}
	\end{flalign}
	\item degrees of freedom
	\begin{flalign}
		n-1
	\end{flalign}
	\item decision rule for rejecting the null hypothesis
	\begin{flalign}
		|T|>t_{n-1,\frac{\alpha}{2}}
	\end{flalign}
\end{itemize}

\fxnote{should be completed when we know if it is right what we are doing}