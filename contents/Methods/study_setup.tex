\chapter{Study setup}

\textit{In this study the peripheral circulation was observed to investigate changes in microcirculation during partial occlusion of blood supply. Infrared imaging was used to measure the temperature changes in the skin of the hand, which was used as an indicator for peripheral circulation.} 


To see if there are changes in the microcirculatory system depending on flow to the observed area, the test was set in two conditions. The first measurement of the hand, was done without intervention, and used as a control measurement. The second measurement of the hand was done during a partial occlusion of the blood supply by a brachial cuff. The partial occlusion of the arm leads to ischemia what leads to a lack of oxygen \cite{martini2012}. Aim of the study was to investigate, if there are measurable changes in the micro temperature oscillations of the skin caused by hypoxia in case of shock. Therefore the induced ischemia was used as a way to mimic hypoxia due to shock.
The duration of a measurement was set to 20 minutes. The lowest frequency of interest is $0.005$ Hz with a cycle time of 3 min 20 s. According to previous studies the recording time was set to 20 minutes to include six cycles of the lowest frequency.\cite{sagaidachnyi2014}
The reason of doing a 50\% restriction of blood flow, was due to the intend of creating ischemia without forcing to much discomfort like pain on test subjects, over the 20 minute occlusion period. Discomfort test was done prior to the start of the experiment.

By first taking the control measurement under normal conditions, the carry-over effect of occlusion was avoided. It enabled taking both measurements of each subject straight successively, what reduced inaccuracies within the setup of both experiments for each subject. The setting was assembled in the Region Hospital Nordjylland.