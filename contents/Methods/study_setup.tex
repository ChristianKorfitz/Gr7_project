\chapter{Study setup}

\textit{In this study the peripheral circulation is observed to investigate changes in microcirculation during partial occlusion of blood supply. Infrared imaging is used to measure the temperature changes in the skin of the hand, which is used as an indicator for peripheral circulation.} 


To see if there are changes in the microcirculatory system depending on flow to the observed area, the test is set in two conditions. The first measurement of the hand, is done under normal conditions, and used as a control measurement. The second measurement of the hand is done during a partial occlusion of the blood supply. The partial occlusion of the arm leads to ischemia what leads to a lack of oxygen \cite{martini2012}. The study is used to investigate, if there are measurable changes in the micro temperature oscillations of the skin caused by hypoxia in case of shock. Therefore the induced ischemia is used as a way to mimic hypoxia due to shock.
The reason to do a 50\% restriction of blood flow, is due to the intend of creating ischemia without forcing to much discomfort on test subjects. Discomfort test was done prior to the start of the experiment.

By first taking the control measurement under normal conditions, the carry-over effect of occlusion is avoided. It enables taking both measurements of each subject straight successively, what reduces inaccuracies within the setup of both experiments for each subject. Therefore a special setting, is assembled in the Regionshospital Nordjylland in Hj\o{}rring.