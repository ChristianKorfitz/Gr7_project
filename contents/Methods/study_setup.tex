\chapter{Study setup}

In this study the peripheral circulation is observed to investigate if there are changes in microcirculation during partial occlusion of blood supply. Infrared imaging is used to measure the temperature changes in the skin of the hand, which is used as an indicator for peripheral circulation. 


To see if there are changes in the microcirculatory system depending on flow levels in the macrocirculatory system, the test is set in two conditions. The first measurement of the hand, which is done under normal conditions, is used as a control measurement. The second measurement of the hand is done during a partial occlusion of the blood supply. The partial occlusion of the arm leads to a lower oxygen supply which leads to ischemia, and is used as a way to mimic sepsis. The reason to do a 50\% restriction of blood, is due to the intend of creating a high level of ischemia without forcing to much discomfort on test subjects.

By first taking the control measurement under normal conditions, the carry-over effect of the occlusion is avoided. It also enables to take both measurements of each subject straight successively, what reduces inaccuracies within the setup of both experiments for each subject. Therefore a special setting, which is shown in \cref{fig:setting} is assembled in the Regionshospital Nordjylland in Hj\o{}rring.