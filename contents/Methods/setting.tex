\chapter{test setting}
In this study the peripheral circulation is observed to investigate if there are changes in microcirculation during partial occlusion of blood supply. Infrared imaging is used to measure the temperature changes in the skin of the hand, what is used as an indicator for peripheral circulation. 
To see if there are changes in the microcirculatory system depending on pressure levels in the macrocirculatory system, the test is set in two conditions. The first measurement of the hand, which is done under normal conditions, is used as a control measurement. The second measurement of the hand is done during a partial occlusion of the blood supply. By fist taking the control measurement under normal conditions, the carry-over effect of the occlusion is avoided. It also enables to take both measurements of each subject straight successively, what reduces inaccuracies within the setup of both experiments for each subject. Therefore a special setting is assembled.

The subject will be placed in a chair with armrests, which allows a good positioning of the measured hand. The measured hand is the dominant hand and must be fixed during the whole test to minimize movement bias. Hence the hand is fixed with double sticking tape on a vacuum pillow which also stables the hand. Obviously the subject has to sit still and is not allowed to move during the test, because the test setting just prevents very small unintentional movements.
XXX cm over the hand the XXX camera is positioned with a tripod. The focus is on XXX .
The camera is via a XXX cable connected with a laptop, which is used to record the measurements with XXX software.

XXX Bild: setting

First of all the camera has to warm up for XXX min. During this laptop, software and all cable connections  should be set in operational readiness. Then the cuff is affixed at the subjects dominant arm without tighten it. After that the subject can take place in the chair and the hand can be fixed on the vacuum pillow with double sticking tape. The vacuum generator is attached to the pillow for giving the hand more stability. Next the camera needs to be positioned XXX cm over the hand with the focus on the region of interest. If the camera is stable, the first measurement can be started for exact 20 min.
Directly after the first measurement the cuff on the arm of the subject is tightened with XXX. This entails a partial occlusion of the blood supply in the measured hand. During the whole procedure the subject is still not allowed to move. Then the second measurement can be started for exact 20 min.

XXX Bild: subject in setting with tightened cuff
