\section{Presentation of the data}

The data is collected with the infrared camera ... ... . The file is an ... fil with the size between 1 kB to two GB.


"Notes" that can become text
Spatial information is the image in the time domain, like when we see it as an image. 
To analyze the data we have we should finde the reagions (27 Andrei will point them out) and find the frequency components of the pixcels of the part of the image. 

filtering of image frequenzy analysis is by a circle with a specific radius that makes the frequency become zero

As we look at digital images we will use the Discrete Fourier Transform (DFT) to analyze the images


A possible approach for the data analysis to be carried out 

- If you have 20 images that are 256x256, how I imagine this working is,

1. Take your 2D FFT or 2D DCT (or blockwise FFT/DCT) for each image
2. Normalize all the values so you can have a good comparison between each image
3. Concatenate all the frequency domain images into a 256 x 256 x 20 matrix
4. Filter out the values that are below a certain threshold because they do not represent a significant component of your image. For example, if the magnitude of the values in the corners that represent high frequency information are very low, then removing them won't make a huge difference to your image because most of the information is elsewhere. But removing those data points create holes in your 3D plot so you can see the information that you're interested in
5. Plot the data in 3D. If you're using Matlab, you can use scatter3(). The plot will be easier to read if you apply a heatmap.

