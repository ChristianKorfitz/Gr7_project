Although there have been several studies, which have dealt with the occurrences within the capillary network, this area is not completely investigated. Especially not the phenomena of the oscillating changes of the vessel diameter of the capillaries. This phenomena, called vasomotion, which occurs in the microcirculatory system is an auto-regulation mechanism that optimizes blood distribution within the microcirculatory system.
With better knowledge about the occurrences during vasomotion, it is possible to improve the treatment of intensive care patients. Particularly for sepsis patients, which are in a high risk of multi organ failure, might be the chance to prevent multi organ failure due to a lack of supply through the microcirculatory system in the organs.

Previous studies detected that the vasomotric blood flow is quantifiable as temperature micro oscillations in the frequency range of $0,005 - 0,05 Hz$. Based on this, it must be possible to detect a difference in the mean spectrum temperature oscillations in the microcirculatory system, depending on the blood flow in the macrocirculatory system.
Therefore a study of vasomotion in the peripheral circulation with infrared thermography is implemented. Thereby the temperature oscillations in the skin, which are used as an indicator for peripheral circulation, are measured by infrared imaging. The aim of this study was to investigate if there are changes in the vasomotric blood flow caused by partial occlusion of the blood supply. The restriction of the blood supply causes an ischemia, which leads to a lack of oxygen. This is used to image one effect of sepsis.
This study lays the foundation for further experiments to elaborate the relationship between the pathological changes in the cardiovascular system and the microcircualtation to draw conclusions out of the vasomotric blood flow to improve the treatment of patients with sepsis.