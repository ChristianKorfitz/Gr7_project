Vasomotion is an autoregulatory mechanism that optimizes blood distribution within the microcirculatory system. Thermal imaging is a promising approach to measure this phenomena.	
Previous studies have detected that vasomotion is quantifiable as temperature micro oscillations in the endothelial (0.005 - 0.02 Hz), neurogenic (0.02 - 0.05 Hz) and myogenic (0.05 - 0.15 Hz) frequency band. Four healthy subjects were recruited to investigate the possibilities of measuring changes in vasomotric activity caused by partial occlusion of blood supply by using thermal imaging.
Measurements were done as a baseline and with 50\% restriction of hand$’$s blood supply by brachial cuff.
No significant difference between the mean magnitudes of baseline and restriction was found. 
Results showed thermal imaging might not be sensitive enough to detect changes in vasomotion and limitations in the experimental setup.