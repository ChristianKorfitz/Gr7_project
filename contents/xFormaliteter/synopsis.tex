Vasomotion is an auto-regulation mechanism that optimizes blood distribution within the microcirculatory system. Thermal imaging is an interesting approach to measure this phenomena.	
Previous studies have detected that vasomotric blood flow is quantifiable as temperature micro oscillations in the frequency range of 0,005 - 0,15 Hz. Four healthy subject was recruited to investigate the possibilities of measuring changes in vasomotric blood flow caused by partial occlusion of blood supply by using thermal imaging.
The temperature oscillations in the skin were measured with an infrared camera and done under normal conditions and with 50\% restriction of hand$’$s blood supply.
No significant difference between the magnitudes of uncuffed and cuffed were found. 
Results showed thermal imaging might not be sensitive enough to detect vasomotion and clear limitations in the experimental setup.