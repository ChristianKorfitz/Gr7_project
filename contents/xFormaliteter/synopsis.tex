
Kronisk obstruktiv lungesygdom er en af de mest dødelige sygdomme, og patienternes livskvalitet forringes gradvist. Det er vist at motion kan forbedre patienternes tilstand, men at mange patienter ikke deltager i rehabilitering. Dette projekt søger derfor en software-relateret løsning, der kan motivere KOL-patienter til fysisk aktivitet i hjemmet. Der tages udgangspunkt i objektorienteret programmering og unified modelling language til udviklingen af softwareløsningen, som forsøger at motivere patienter til højere aktivitetsniveau ved brug af en sofacykel, hvor rehabiliteringsgrupperne i fællesskab bidrager til en gruppedistance. Det udviklede system indeholder en applikation, som kan vise patientens og rehabiliteringsgruppens aktivitet, en server indeholdende alle patientinformationer, samt et hardwaremodul til optagelse af aktivitet foretaget på sofacyklen. Systemet tilbyder derigennem, at KOL-patienter kan udføre fysisk aktivitet udover rehabiliteringsforløbene, og at sundhedspersonale kan monitorere deres aktivitet. %Videre udvikling vil potentielt indeholde muligheden for konkurrence mellem rehabiliteringsgrupperne for højere motivationsfaktor, udvidelse af brugeradministration samt kompensation for den daglige variation af patienternes tilstand.