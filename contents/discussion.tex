\chapter{Discussion}

The present study used thermal imaging to detect if there are measurable changes in the micro temperature oscillations of skin depending on blood flow to the observed area of the hand. Whereby the temperature changes in the skin were used as indicator for vasomotric activity. %vorgelegte arbeit beschreiben
The means of the regarded frequency bands in both conditions resemble one another. Hence the results of the paired t-test of the regarded frequency band means show no clear significance.  %kurze zsmfsg der ergebnisse

\subsubsection{Reasons for insignificant results}
The obvious assumption that there are no changes in the microcirculatory system or rather in vasomotion by 50\% restriction of blood flow can be substantiated either by independence of the microcirculatory system because the available amount of blood in the microcirculatory system is dependent on the amount of blood the macrocirculatory system provides which means that the microcirculatory system is not independent. Consequently this explanation can be excluded. 
Another explanation for the insignificant result would be by incorrect brachial occlusion. Incorrect occlusion %means the used cuff is not able to provide
would yield insufficient restriction for affecting the microcirculatory system. %Firstly it can be due to the shape of the cuff what is unlikely because a standard cuff which dose not differ from other cuffs was used. 
A wrong occlusion pressure could be the cause of this. As it can be seen in table \ref{chap:protocol}, the blood pressure measurements of three subjects delivered high values around $140 mmHg$ and around $150 mmHg$. Since those three subject were in different age and shape, a suspicion for incorrect values arise. %But regarding the other subject's value which is in a low normal range around $100 mmHg$ and the fact that the blood pressure measurements were conducted by an anesthesiologist with a blood pressure monitor used in clinical routine dash this suspicion.
Even if the blood pressure monitor delivered wrongly high values, the outcome of the second measurement would not have been influenced negatively. This would only lead to a calculated occlusion pressure higher than the one needed to reach the intended restriction, which would just lead to a larger difference between both conditions. It would be more problematic with a occlusion level to low.  %lack of restriction

The small sample size of four subjects applied to the statistical methods are not sufficiently meaningful. With a larger amount of subjects this study would get a more meaningful result. A larger amount of subjects might therefore have provided a significant difference between both conditions. An even gender distribution might also have been preferable. But regarding the measured data there cannot be seen a clear difference between female and male subjects.
Furthermore the missing randomization of the study design can be criticized. Though in the case of this study it was not possible to randomize which measurement is conducted first. If the condition with 50\% restriction of the blood flow would have been carried out first, the data of the measurement under normal conditions would have been affected by a carry over effect. A randomization could possibly have been carried out by measuring one condition one day and the other the next day, and made sure when picking out ROI that it is the same locations on the hand in the data processing part. It should also be considered if there is an effect of the hand just lying still. For instance subject 2 and 3 had cold hands in the beginning of the experiment and got progressively warmer as the experiment went on. There might be a need for an habitation period for the hand, but this would have to be investigated further. 
However this study was used as a pilot study to investigate if thermal imaging can be used to detect vasomotric activity. Also impaired by subsequent elaborated limitations it would not have been worthwhile to recruit a larger amount of subjects even though it was the first intention.
%Regarding the previous studies, which were able to measure vasomotric activity and the changes of the spectral components influenced of diseases, let assume that the results are due to a lack in the methodology of this study.

\subsubsection{Limitations in the setup}
%The thermal camera contains sensitive  microbolometers which from time to time needs to recalibrate. 
The used thermal camera of uncooled type induced jumps and pixel drift within the intervals between each jump. Therefore a correction method was implemented, which clearly affects parts of the signal in specific ROIs. The pixel drift increases with increasing distance to the center of the thermal image why pixel drift of ROIs located in the outer areas of the thermal image cannot be completely compensated for with the implemented correction method. The explanation for this is likely that the correction method is based on the assumption that the drift component in every interval is linear. This assumption might be wrong and a correction method that uses another regression method or combines different regression methods might adjust the artifacts in a better way. As a result the ROIs located in the outer area of the thermal image were excluded from the data analysis. Another camera might also be preferable to reduce the risks of technical artifacts like these. 

Vasomotric activities measured by other studies where done in the fingertips \cite{sagaidaclhnti2012a, sagaidachnyi2014}. Whereas this study compared just areas on the back of the hand based on the assumption that vasomotric activity can be measured equally in every area of hands skin. The results lead to the suspicion that vasomotric activity cannot be measured equally on the back of the hand and the fingertips. The reason for not taking the fingertips into account is the uselessness of the data in these areas. Key reasons are the properties of the thermal camera. Since the thermal pixel drift is known the regions of most interest should be centered in the thermal image. 

Furthermore the smaller the observed area the closer the thermal camera has to be to this area. Thus the area represented by one pixel gets smaller. Within this study one pixel represents an area with a diameter around $417µm$. Whereas previous studies observed smaller areas which means that the pixel represent a smaller area. With the use of a larger ROI, it might be that the amount of inverse dilating capillaries is equal and thereby canceling each other out. Due to inverse dilating capillaries there might be no changes over time measurable because the frequency contents are occurring alternating in the different capillaries. In addition, it is obvious that the artifacts content in the signal is significantly higher than in the temperature signals detected in previous studies. Comparing the signals the question arises if the artifacts overlaps or suppresses the frequency content of vasomotric activity. Even though temperature changes over time in the skin were detected, it is uncertain, if this signal just represents the general skin temperature or also vasomotric activity. %is there even a signal?

Other limitations might be that the subjects hands had to be stabilized by a vacuum pillow with the assumption the subjects were still, but this does not give sufficient support to inhibit every movements of the subject.  A mechanism for better stabilization of the hands should be included. An approach to compensate for instability of the hand during the experiment could have been to include image alignment in the data processing to limit the drawbacks of possible movements in the recordings. Furthermore with the used setup exact same conditions for each subject cannot be granted. For instance the room temperature should have been measured beforehand and taken into consideration before the start of each experiment to create the same conditions for all subjects. The presetting of the room temperature might have affected the data. This suspicion is enhanced by regarding subject 1's hand temperature which was $38^\circ C$. Also by regarding the hand temperature of $26^\circ C$ of subjects 2 and 3, who had cold hands during the measurement, this suspicion is enhanced. The software settings should be verified and if necessary changed beforehand of the experiment.

An optimization of the test setting would require a more controlled setup of the experiment and a thermal camera of higher quality preferable of a cooled type would have improved the validity of this study.