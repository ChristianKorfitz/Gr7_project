\chapter{Discussion}

The present study used thermal imaging to detect if there are measurable changes in the micro temperature oscillations of skin depending on blood flow to the observed area. Whereby the temperature changes in the skin were used as indicator for vasomotric activity. %vorgelegte arbeit beschreiben

The means of the regarded frequency bands in both conditions resemble one another. Hence the results of the paired t-test of the regarded frequency band means show no clear significance. Likewise the qualitative comparison by mapping of t-values shows no significant difference. %kurze zsmfsg der ergebnisse
\\
\\
The obvious assumption that there are no changes in the microcirculatory system or rather in vasomotion by 50\% restriction of blood flow can be substantiated either by independence of the microcirculatory system or by incorrect occlusion of the arm.

The available amount of blood in the microcirculatory system is depending on the amount of blood the macrocirculatory system provides what means that the microcirculatory system is not independent. Consequently this explanation can be excluded.

Incorrect occlusion of the arm means the cuff which was used for occluding the arm during the second measurement is not able to provide sufficient restriction for affecting the microcirculatory system. Firstly it can be due to the shape of the cuff what is unlikely because a standard cuff which dose not differ from other cuffs was used. Secondly a wrong occlusion pressure could be applied. As it can be seen in TABLE, the blood pressure measurements of three subjects delivered high values around $140 mmHg$ and around $150 mmHg$. Since those three subject were in different age and shape, the suspicion arises that the measured values are incorrect. But regarding the other subject's value which is in a low normal range around $100 mmHg$ and the fact that the blood pressure measurements were conducted by an anesthesiologist with a blood pressure monitor used in clinical routine dash this suspicion.
Even if the blood pressure monitor delivered wrongly high values, the outcome of the second measurement would not have been influenced negatively. A wrongly high occlusion pressure leads to a calculated occlusion pressure which is higher than needed to reach the intended restriction. At worst the restriction would have been higher than 50 \% of the blood flow, what is just leading to a larger difference between both condition. This would have affected the outcome not negative. %lack of restriction
\\
\\
The small amount of subjects must not be disregarded. With a sample size of four subjects applied statistical methods are not sufficiently meaningful. With a larger amount of subjects this study would get a more meaningful result. It might be possible that a larger amount of subjects provides a significant difference between both conditions.
Also the gender distribution of the subjects was not equal. It might be better to use the same amount of male and female subjects. But regarding the measured data there cannot be seen a difference between female and male subjects.

Furthermore the missing randomization of the study design can be criticized. Though in the case of this study it was not possible to randomize which measurement is conducted first. If the condition with 50\% restriction of the blood flow would have been carried out first, the data of the measurement under normal conditions would have been affected by a carry over effect.

However this study was used as a pilot study to investigate if thermal imaging can be used to detect vasomotric activity. Also impaired by subsequent elaborated limitations it would not have been worthwhile to recruit a larger amount of subjects even though it was the first intention.
\\
\\
Regarding the previous studies, which were able to measure vasomotric activity and the changes of the spectral components influenced of diseases, let assume that the results are due to a lack in the methodology of this study.

Those vasomotric activities were measured in the fingertips. Whereas this study compared just areas on the back of the hand based on the assumption that vasomotric activity can be measured equally in every area of hand's skin. The results lead to the suspicion that vasomotric activity cannot be measured equally on the back of the hand and the fingertips.

The reason for not taking the fingertips into account is the uselessness of the data in these areas. Key reasons are the properties of the Gobi 640 17µm GigE camera (Xenics NV, Belgium). This uncooled thermal camera contains sensitive  microbolometers which need from time to time a recalibration. Hence occurring jumps and the pixel drift within the intervals between the jumps make the data unusable. Therefore a correction method was implemented, which clearly affects parts of the signal.

The correction method is based on the assumption that the drift component in every interval is linear. This assumption might be wrong and a correction method that uses an other regression method or combines different regression methods adjust the artifacts in a better way. 

Besides, the pixel drift increases with increasing distance to the center of the thermal image. Regarding the ROIs located in the outer areas of the thermal image the pixel drift cannot be completely compensated of the implemented correction method. As a result the ROIs located in the outer area of the thermal image were excluded from the data analysis. %why not fingertips

Furthermore the smaller the observed area the closer the thermal camera has to be to this area. Thus the area represented by one pixel gets smaller. Within this study one pixel represents an area with a diameter around $417µm$. Whereas previous studies observed smaller areas what means that the pixel represent a smaller area.
It might be that previous studies detected vasomotric activity by using smaller ROIs. With the use of a larger ROI, it might be that the amount of inverse dilating capillaries is equal. Due to inverse dilating capillaries there might be no changes over time measurable because the frequency contents are occurring alternating in the different capillaries.

In addition, it is obvious that the noise content in the signal is significantly higher than in the temperature signals detected in previous studies. Comparing the signals the question arises if the noise overlaps or suppresses the frequency content of vasomotric activity. Even though temperature changes over time in the skin were detected, it is uncertain, if this signal just represents the general skin temperature or also vasomotric activity. %is there even a signal?
\\
\\
As outlined the Gobi 640 17µm GigE (Xenics NV, Belgium) thermal camera might be unsuitable in this setup. Besides, this was not the only limitation of the setup.

The subjects hands were stabilized by a vacuum pillow with the assumption the subjects were still, but it gives not sufficient support to inhibit all movements of a subject. Furthermore with the used setup exact same conditions for each subject cannot be granted. The position of the hand and the position of the camera varies marginally between the subjects due to missing markings. There was no possibility to influence the room temperature or to measure it and take it into consideration. So the room temperature was preset to $25^\circ C$ for all subjects. Equally the software settings were preset and could not be influenced.

The presetting of the room temperature might have affected the data. This suspicion is enhanced by regarding subject 1's hand temperature which was $38^\circ C$. Also by regarding the hand temperature of $26^\circ C$ of subjects 2 and 3, who had cold hands during the measurement, this suspicion is enhanced. %warum unser setup kacke ist
\\
\\
Despite the mentioned limitations of the setup there are some points which can be improved using the provided materials and given conditions. The software settings should be scrutinized and if necessary changed. Also the room temperature needs to be measured and taken into consideration by entering into the software setting. Since the thermal pixel drift is known the alleged best regions should be centered in the thermal image and the thermal camera should put closer to the observed ares.
Moreover a markings for the placement of every component provide more stability of the setup between the subjects and increase the reproducibility. Nevertheless an improved setup could not compensate for limitations given by the used thermal camera. An optimal test setting requires a thermal camera of higher quality preferred a cooled camera. In addition a mechanism for better stabilization of the hands should be included. To create the same conditions for each subject, the room temperature should not vary and be adjustable.