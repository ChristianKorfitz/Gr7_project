\section{Techniques of measuring vasomotion} \label{freq}

For some time it has been the interest of scientists and health care professionals to get a better understanding of the mechanisms that control and regulate local blood flow in the microcirculatory system\cite{sagaidachnyi2014,sagaidachnyi2017,geyer2004,liu2012}. 
Visualization of the vessels in the skin and the way these behave can be important for assessment of stages of sepsis as mentioned before in \cref{chap:sepsis}, but also in peripheral vascular disease, the results of skin reconstructive surgery, wound and ulcer management.\cite{liu2012,kanta2014}

For measuring regulation in the peripheral blood flow, it is assumed that these oscillating changes are the source of thermal waves propagating from microvessels toward the skin’s surface. Especially thermal imaging uses this concept.\cite{sagaidachnyi2017}
Furthermore a correlation between skin temperature in fingertips and blood flow oscillations has been found\cite{sagaidachnyi2014}.

There are multiple different techniques of measuring blood flow in the peripheral circulatory system. Some of these are: capillaroscopy, laser Doppler flowmetry (LDF), laser speckle contrast and orthogonal polarisation spectral imaging. These have been used differently trying to quantify functional aspects of skin vasculature.\cite{liu2012} With Laser Doppler flowmetry being one of the most used\cite{geyer2004} and thermal imaging being the new technique of measuring vasoregulation, these will be further exploited\cite{sagaidachnyi2014}.     

\textbf{Thermal imaging}

In studies made by a Russian group lead by Sagaidachnyi et al. the use thermal imaging has been used to study vasomotion. In their studies they seek to get better understanding of the relationship between blood flow oscillations and temperature oscillations, and see if it was possible to recreate the flow oscillation from temperature recording. Recordings of flow where done by Photoplethysmography and temperature of the skin by thermal imaging. The recordings were made on a small point of the fingertip. Trough their work, five frequency bands were identified as vasomotion activity, and are following: endothelial (0.005–0.02 Hz), neurogenic (0.02-0.05 Hz), myogenic (0.05-0.15 Hz), respiratory origin (0.15-0.4 Hz) and cardiac origin (0.4-2.0 Hz).\cite{sagaidachnyi2017,sagaidachnyi2014}
The choice of using thermal imaging to study vasomotion comes with some advantages. Mainly a larger sample area, but also a higher temporal up to 105 fps and spatial resolution 2048 × 1536 pixels. In addition being a non invasive way of measuring vasomotion is to be taken in to consideration.\cite{sagaidachnyi2017}



\textbf{Laser Doppler flowmetry}

In an other study from Geyer et al. vasomotion is investigated trough the use laser Doppler flowmetry as recording technique. In the study vasoregulation variables are sought quantified. LDF is a non invasive approach to measuring changes in vasomotion. The technique register changes in the depth of 1 mm, and works like Doppler ultrasound, utilizing the shift in frequency, but instead of ultrasound, it uses light reflected from red blood cells. This study found the same frequency bands as Sagaidachnyi et al. with minimal difference. Data obtained were analyzed trough spectral analysis. Wavelet transform was used as method instead of the most used fourier, because wavelet analysis offered better resolution to reveal characteristics in the low frequency area.\cite{geyer2004}
LDF uses a small sample area and the laser probe allows a sampling area as small as 1 mm$^3$.\cite{brothers2010} 

  
\section{Summarizing}

Both Geyer et al. and Sagaidachnyi2017 et al. managed to show spectral components relating to vasomotion. The techniques both uses an non invasive approach, even though the methods are different when measuring red blood cell count compared to temperature. The use of thermal imaging as the method of measuring vasomotion offers interesting opportunities. Larger sampling area would allow interpretation and study of a more global tissue area. Along with the resolution of thermal imaging cameras, this makes thermal imaging the choice of measuring technique to be used in this study. 


%Spectral components of the vasomotion is assumed to be changed under influence the some diseases.   For instance, the depression in the amplitude of endothelial blood flow oscillations is considered to be an indicator of endothelial dysfunction and a precursor of many kinds of cardiovascular disorders such as arterial hypertension, cardiac ischemia, and atherosclerosis. The increasing amplitude of blood flow oscillation within the neurogenic frequency band is characterized by a decrease
%of vascular resistance and an increase of volume blood flow through the arteriovenous shunt.\cite{Sagaidachnyi2017}