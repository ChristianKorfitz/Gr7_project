\chapter{Hemodynamics}
\textit{Hemodynamics explains the movement or flow of blood. It is influenced by parameters like blood pressure, blood volume, cardiac output, blood composition, etc. It is possible to measure some of the hemodynamic parameters non-invasive.\cite{martini2012,thiriet2008}}

\section{Physiological Base}
The blood pressure is regulated by baroreceptors in the walls of the big arteries in chest and neck area. These receptors register the changes of the elongation of the vessels and transmit this information to medulla oblongata. With the received pressure information  initiates the medulla oblongata, if necessary, regulatory measures. The sympathicus is responsible for the short-therm regulation. Both, middle-term and long-term regulation, is made by the kidneys. For middle-term regulation messenger substances are released, which entail vasoconstriction. The long-term regulation occurs  per pressure diuresis or reabsorption in the kidneys. It is possible to measure different blood pressures at different places in the cardiovascular system, for example the mean arterial pressure $(MAP)$. The $ MAP $ increases in relation to the stroke volume and decreases when blood flows into the peripheral system.\cite{martini2012,thiriet2008} \\

The cardiac output $ (CO) $ states the blood volume, which is pumped by the heart per time unit $(HR)$. The calculation of the $ CO $ as follows.\cite{martini2012}
\begin{flalign}
	CO=HR\times stroke volume
\end{flalign}


\section{Physical Base}
To consider the hemodynamics, it is possible to draw similarities by analogy of physical laws. Especially of Ohm's law $ R=\frac{U}{I} $ or rather $ I=\frac{U}{R} $. A special case of Ohm's law constitutes Hagen-Poiseuille's law in the field of fluid dynamic and rheology. Hagen-Poiseuille's law describes the laminar flow of an homogeneous Newtonian fluid through a rigid pipe depending on characteristics of the fluid and of the pipe.\cite{thiriet2008,noordergraaf2011}

Blood is an inhomogeneous suspension of liquid and corpuscular components, whose viscosity $ \eta $ depends on more factors than the temperature, and is consequently no Newtonian fluid. Nevertheless it is possible to draw conclusions by analogy of Hagen-Poiseuille's law for the computation of the hemodynamics.\cite{thiriet2008,noordergraaf2011}
\begin{flalign}
	\frac{V}{t}=\frac{r^{4}\times\pi\times\Delta P}{8\times\eta\times I}
\end{flalign}

Here is the volume flow equivalent to the electrical current $ I $ and the pressure difference $ \Delta $P to the electric voltage $ U $. Thus, the calculation of the resistance as follows.\cite{thiriet2008,noordergraaf2011}
\begin{flalign}
	R=\frac{8\times I\times\eta}{r^{4}}
\end{flalign}

Thereby volume flow increases 16 times and the resistance decreases 16 times for double radius $ r $.\cite{thiriet2008,noordergraaf2011}\\
