

\section{Macrocirculatory system}
The main function of the cardiovascular system is the blood supply of the whole body and the transportation of metabolites. The propulsion of this is the heart. It generates the systolic blood pressure through the strength of the left ventricle. The pressure difference between the heart and the periphery emerging from there, ensures the blood flow. The blood flows from regions with high pressure, like the aorta, to regions with low pressure, like the periphery.\cite{martini2012}

The heart supplies the body with blood through the systemic and the pulmonary circuit. The heart regulates the blood allocation with adjustment of stroke volume and heart frequency. The oxygen-rich blood accumulates in the left ventricle, from where the blood is pushed out through the aortic valve into the aorta and spread via the arteries into the whole body. The venous system returns the de-oxygenated blood back to the heart into the right atrium. From there the blood flows into the right ventricle and is pushed out through the pulmonary valve into the lung arteries, where gas exchange of the blood occurs. Subsequent the oxygen-rich blood flows via the pulmonary veins back to the left heart to supply the body.\cite{martini2012}

As mentioned, there are two types of vessels, arteries and veins. The difference between the two types of vessels is that arteries transport the blood away from the heart and veins transport blood to the heart. There are also differences in the structure of arteries and veins.
Arteries consist of three different layers, tunica interna, tunica media and tunica externa. The tunica interna consists of vascular endothelium, the tunica media consists of smooth muscle cells and elastic fibres, the tunica externa consists of connective tissue and also elastic fibres. Furthermore, there are two different types of arterial vessels. In arteries of the elastic type prevail the elastic fibres in the tunica media. This allows an abrupt extension of the vessel during the systole and ensuing constriction, due to this the blood is transported. In arteries of the muscular type prevail the muscular fibres in the tunica media. This allows regulation of the lumen by constriction and dilatation, whereby the resistance and the blood flow in the organs is regulated.\cite{martini2012}

Venous vessels are similarly structured like arterial vessels, however they are thinner and have semilunar valves inside, to inhibit back flow inside the vessels. This system is supported by the skeletal muscles which help to hold up blood flow. The arterial and the venous vessel system are connected through the capillary system in the microcirculatory system.\cite{martini2012}
