\chapter{Anatomy and Physiology}
Top coming...

\section{Macrovascular system}
The main function of the cardiovascular system is the blood supply of the whole body and the transportation of metabolites. The propulsion of this is the heart. It generates through the sputum strength of the left ventricle the systolic blood pressure. The pressure difference between the heart and the periphery emerging there from, ensures the blood flow. The blood flows from regions with high pressure, like the aorta, to regions with low pressure, like the periphery.

The heart supplies the body through two circuits with blood. On this occasion the heart regulates the blood allocation with adjustment of stroke volume and heart frequency.

The oxygen-rich blood accumulates in the left ventricle. From there the blood is thrown out through the aortic valve into the aorta and via the arteries spread into the whole body. The venous system returns the meanwhile low in oxygen blood back to the heart into the right atrium. From there the blood flows into the right ventricle and is thrown out through the pulmonary valve into the lung arteries. In the lung happens the gas exchange of the blood. Subsequent the oxygen-rich blood flows via the pulmonary veins back to the left heart to supply the body.

As mentioned, there are two types of vessels, arteries and veins. The difference between those two types of vessels is on the one hand that arteries transport the blood away from the heart and veins solely transport blood to the heart. On the other hand, there are some differences in the structure of arteries and veins.
Arteries consist of three different layers, tunica interna, tunica media and tunica externa. The tunica interna consists of vascular endothelium,  the tunica media consists of smooth muscle cells and elastic fibres, the tunica externa consists of connective tissue and also elastic fibres.

Furthermore, there are two different types of arterial vessels. In arteries of the elastic type prevail the elastic fibres in the tunica media. This allows an abrupt extension of the vessel during the systole and ensuing constriction, due to this the blood is transported. This phenomena is called windkessel function. In arteries of the muscular type prevail the muscular fibres in the tunica media. This allows regulation of the lumen by constriction and dilatation, whereby the resistance and the blood flow in the organs is regulated.

Venous vessels are similarly structured like arterial vessels, however they are thinner and have also semilunar valves inside, to inhibit back flow inside the vessels. This system is supported by muscle pump.

The arterial and the venous vessel system are connected through the capillary system. [martini2012]
