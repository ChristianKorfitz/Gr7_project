\chapter{Vasomotion in disease}
\textit{This chapter describes pathologic incidents in the cardiovascular system and organs during shock. This is done with the interest of finding physiological parameters that indicate shock development.}

\section{Types of shock}
In general, shock is characterized by hypoxia in tissues due to inadequate blood supply. The hypoxia during a shock leads to the deposition of metabolisms in organs what results in an increased risk of multi organ dysfunction. 
There are four different types of shocks:\cite{lauridsen2015,vincent2013}
\begin{itemize}
	\item \textbf{Hypovolaemic shock} is caused by a lack of volume. Either as a consequence of blood loss (hemorrhagic shock) or of water, plasma or electrolyte loss.
	\item \textbf{Cardiogenic shock} is caused by cardiac failure, for instance myocarditis, cardiomyopathy in final stage or acute myocardial infarction.
	\item \textbf{Obstructive shock} is caused by obstruction of blood flow, for instance pulmonary embolism, cardiac tamponade or tension pneumothorax 
	\item \textbf{Distributive shock} covers, inter alia, septic shock, anaphylactic shock and neurogenic shock
\end{itemize}

Main cause for cardiogenic, hypovolaemic and obstructive shock is a decreased cardiac output without adapting the peripheral resistance. This leads to a lack of oxygen supply.
Whereas the main cause for a distributive shock lies in a dysfunction of the peripheral areas in terms of reduced systemic vascular resistance as well as varied oxygen extraction.\cite{vincent2013}

Shock affects alterations in the microcirculatory system and interferes perfusion\cite{maier2013}. The changes in the circulatory system are in the following section further elaborated with the aid of sepsis.