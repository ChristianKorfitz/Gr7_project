\chapter{Vasomotion in disease}
This chapter describes pathologic incidents in the cardiovascular system and organs during shock.

In general shock is characterized by hypoxia in tissues due to inadequate blood supply. The hypoxia during a shock leads to the deposition of metabolisms in organs what results in a increased risk of multi organ dysfunction. 
There are fout different types of shocks: \cite{lauridsen2015;vincent2013}
\begin{itemize}
	\item hypovolaemic shock caused by a lack of volume
	\begin{itemize}
		\item blood loss = hemorrhagic shock
		\item water, plasma or electrolyte loss
	\end{itemize}
	\item cardiogenic shock caused by cardiac failure
	\begin{itemize}
		\item myocarditis
		\item cardiomyopathy in final stage
		\item acute myocardial infarction
	\end{itemize}
	\item obstructive shock caused by obstruction of blood flow
	\begin{itemize}
		\item pulmonary embolism
		\item cardiac tamponade
		\item tension pneumothorax
	\end{itemize} 
	\item distributive shock
	\begin{itemize}
		\item septic shock
		\item anaphylactic shock
		\item neurogenic shock
	\end{itemize}
\end{itemize}

Main cause for cardiogenic, hypovolaemic and obstructive shock is a decreased cardiac output without adapting the peripheral resistance. That leads to a lack of oxygen supply.
Whereas the main cause for a distributive shock lies in a dysfunction of the peripheral areas in terms of reduced systemic vascular resistance as well as varied oxygen extraction. \cite{vincent2013}

Shock affects alterations in the microcirculatory system what interferes the perfusion \cite{maier2012}. The changes in the circulatory system are in the following part further elaborated with the aid of sepsis.