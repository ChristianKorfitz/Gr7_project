\section{Statistical approach}
As the actual distribution of the data is unknown, the statistical approach is based on the assumption that the data is normal distributed due to natural variance within subjects. 

To test if there is a statistical significant difference between the uncuffed and cuffed condition a paired t-test is applied on the outcome data from the scalograms. The approach builds on using the total mean of each frequency band, condensing these down to just one number for each frequency band for one ROI for one subject, the approach of the calculation is explained in \ref{sec:scalogram}. 
Before a paired t-test is computed for the data, a boxplot is of the data is shown to get a visual representation of the data of the three frequency bands. 

A paired t-test will be performed on all five regions, where uncuffed is the before condition and cuffed is the after condition. The outcome of the paired t-test will either be rejecting the h0 hypothesis, that there is no difference between the two conditions, or not rejecting the h0 hypothesis, which indicates that there is a difference.

The significance level of 0.005 is used the statistical test. This measures the likelihood of rejecting $h_0$ when it is true, which will lead to a type 1 error. The 0.005 significance level presumes that there is a $5\%$ risk of concluding the sample mean to fall withing the critical region at the level of 0.005 and thereby concluding the wrong assumption. \cite{zar2014}