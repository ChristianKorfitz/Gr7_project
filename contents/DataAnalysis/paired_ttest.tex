\section{Statistical approach}
As the actual distribution of the data is unknown, the statistical approach was based on the assumption that the data is normal distributed due to natural variance within subjects. 

It is chosen to test for significant difference between the condition by using a paired t-test on the mean magnitude for a frequency band. Using the mean magnitude as feature for the statistical test build on an approach presented in Liu et al.\cite{liu2015} 
To prepare the data for the paired t-test, first the average magnitudes for each time period of endothelial, neurogenic and myogenic frequency band were calculated by equation \ref{eq:W_avg}.
\begin{flalign}
	W(n_{f})=\frac{1}{N} \sum_{n=1}^{N} W_n(n_{f})
	\label{eq:W_avg}
\end{flalign}
Where N denotes the total number of elements in the frequency band, $W$ is the magnitude of the wavelet, $n_{f}$ is the respective frame and $n$ is the current element of the magnitude of the frequency band.

Then the mean of the average magnitude over the time period is calculated by equation \ref{eq:W_mean}.
\begin{flalign}
	W_{mean}=\frac{1}{N_{f}} \sum_{n_f=1}^{N_{f}} W(n_{f})
	\label{eq:W_mean}
\end{flalign}
Where $W_{mean}$ denotes the mean value of the frequency band over the time period and $N_{f}$ is the total number of frames.
This gives a single value for the specific frequency band to use in the statistical test for comparison between the two conditions.

Before a paired t-test was computed, a box plot was shown to get a visual representation of the mean magnitudes within the three frequency bands. 

A paired t-test was performed on all five regions. A significance level of 0.05 was used the statistical test.\cite{zar2014} 
%This measures the likelihood of rejecting null-hypothesis when it is true, which will lead to a type 1 error. The 0.005 significance level presumes that there is a $5\%$ risk of concluding the sample mean to fall within the critical region at the level of 0.05 and thereby concluding the wrong assumption. 