\section{Reason for unexpected data}


Because of the data seen in figure xx in section zz, a real representation the natural occurrence, and therefore it is assumed that the greater shifts is due to a technological limit. Because of this assumption it is chosen to further investigate the buildup of thermal cameras and look into other papers, to see if they have meet same difficulties.

\subsection{Thermal pixel drift}

Thermal cameras are composed of a matrix of microbolometers as mentioned in section \textbf{make ref to physics}. Each microbolometer is also known as a pixel detector for thermal radiation.\cite{olbrycht2015,wolf2016} Unfortunately is shows that these microbolometers are really sensitive to noise especially in uncooled cameras. The noise comes from 



In a study by Eriksen et al. where thermal imaging was assessed as use for measuring temperature of electrical systems, data recording showed similar behavior as recording in this study. They clearly state that two types of noise is present in their recording. One being white noise from the radiation detector and electronics, and another being a low frequency technical noise. \cite{eriksen2014}