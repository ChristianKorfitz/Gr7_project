\section{Reason for unexpected data}


Because of the data seen in figure xx in section zz, a real representation the natural occurrence, and therefore it is assumed that the greater shifts is due to a technological limit. Because of this assumption it is chosen to further investigate the buildup of thermal cameras and look into other papers, to see if they have meet same difficulties.

\subsection{Thermal pixel drift}

In a study by Eriksen et al. where thermal imaging was assessed as use for measuring temperature of electrical systems, data recording showed similar behavior as recording in this study. They clearly state that two types of noise is present in their recording. One being white noise from the radiation detector and electronics, and another being a low frequency technical noise. To compensate for these artifacts, a moving average filter was applied.\cite{eriksen2014}


Thermal cameras are composed of a matrix of microbolometers as mentioned in section \cref{sec:cam}. Each microbolometer is also known as a pixel detector for thermal radiation.\cite{olbrycht2015,wolf2016} Unfortunately it shows that these microbolometers are really sensitive to noise especially in uncooled cameras. The noise is formed because each microbolometer has a different response to the same infrared excitation. Furthermore it is assumed by some that this response changes linear\cite{olbrycht2015}. This drift in each microbolometer in the focal plane array is also known as non-uniformity. To achieve radiometric precision the camera has to make a correction for this drift called non-uniformity correction. A common way to recalibrate bolometers is to move a shutter in between the lens and the focal plane array. The shutter has the same color and by statistical calculation it is possible to find the drift for each bolometer and create a new base.\cite{olbrycht2015,wolf2016} This auto-adjustments might be what Eriksen et al. saw in their recordings. 




