\chapter{Preparation of data}

\textit{The following chapter will present how the data files from the thermal camera was converted into a temperature traces. Furthermore the chosen regions of interest and applied correction method will be presented.}  

\section{Dividing recording into frames}

Data acquired from the thermal camera using the Xeneth 2.6 software was saved as an xvi file. These files were read into MATLAB R2017b as an uint16 vector file. Before the frames could be separated from the xvi file, the header in front of the files needed to be excluded. The header contained 307729 data points. With the header removed, the frame separation could be carried out. This was done by first calculating the size of one frame. When knowing that each frame would have the dimensions of 480 x 640 pixels, the size of one frame would correspond to 307200 data points for each frame. It should be considered that each frame also contained a 16 bit header. The number of frames in one recording was calculated by dividing the length of one frame by the entire length of the data file containing all frames, without the file header. The data points for each frame were trimmed for its specific frame header and reshaped from a vector into a matrix and verified by showing the images. An example of a raw image can be seen on figure \ref{fig:hand}
The images contain the pixel intensities of values from 0 to 65535, which is corespondent to the size of the uint16 bit file. 
% or if it is the int16 bit file it's values from -32768 to 32767. But the uint16 image looks better and is easier to see, so i think we should use thatone - it won't affect the dataanalysis. 

% Maybe write about temperature convertion

\begin{figure}[H]
	\includegraphics[width=0.6\textwidth]{figures/uint16Hand}  %<--but is not needed.
	\caption{Image of one frame from subject 1 after separation.}
	\label{fig:hand}  %<--give the figure a label, so you can reference!
\end{figure}