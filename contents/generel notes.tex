

\section{Notes}

Oscillations in skin blood flow are the source of thermal waves
propagating from micro-vessels toward the skin’s surface, as assumed in
this study.

The developed technique was verified within
0.005–0.1 Hz, including the endothelial, neurogenic and myogenic frequency
bands of blood flow oscillations. 

The prospective
applications of the thermography-based blood flow imaging technique include
non-contact monitoring of the blood supply during engraftment of skin flaps
and burns healing, as well the use of contact temperature sensors to monitor
low-frequency oscillations of peripheral blood flow.

Instruments used for the imaging of blood flow in skin are currently being applied to the
diagnosis and treatment of peripheral vascular diseases, hypertension and diabetes, as well
as being utilised during healing of wounds, burns and chilblains Human skin microcirculation
is used to investigate vascular mechanisms in a variety of disease.

Modern skin blood flow imaging techniques such as laser Doppler perfusion imaging
(LDPI), laser speckle contrast analysis (LASCA), imaging photoplethysmography (iPPG)
and nailfold capillaroscopy mainly work in the visible wavelength range (

The main advantages of thermography
systems include their ability to capture larger skin surface areas, their high temporal
(up to 1 · 105 fps) and spatial (2048 × 1536 pixels) resolutions, the image stability across
observation angles, the lack of an external emitter, and their facility of displaying results in
degrees, i.e. one of the basic component of the International System of Units (SI).
Currently, thermographic investigations of haemodynamic phenomena are focused on two
aspects. The first aspect is associated with the imaging of venous or arterial networks without
quantitative blood flow assessment. In this, the imaging technique can include records of the
temperature response to brachial cuff occlusion (Liu et al 2012), the temperature response to
periodic blood flow modulation by means of the brachial cuff (lock-in thermography) (Wu
et al 1996, Bouzida et al 2008, 2009) and the response to external heating (Harrison and



vessels varies due to the activities of different physiological subsystems (Salerud et al
1983). Five frequency bands have been identified (0.005–0.02 Hz, 0.02–0.05 Hz, 0.05–0.15
Hz, 0.15–0.4 Hz, and 0.4–2.0 Hz), corresponding to endothelial (metabolic), neurogenic,
myogenic, respiratory, and cardiac origins (Geyer et al 2004). The low-frequency blood flow
oscillations play a significant diagnostic role owing to their higher amplitude. For instance,
the depression in the amplitude of endothelial blood flow oscillations is considered to be an
indicator of endothelial dysfunction and a precursor of many kinds of cardiovascular disorders
such as arterial hypertension, cardiac ischemia, and atherosclerosis. The increasing amplitude
of blood flow oscillation within the neurogenic frequency band is characterized by a decrease
of vascular resistance and an increase of volume blood flow through the arteriovenous shunt.

Thus, we
assume that the heat flux from a deeper part of the skin to its surface in the z direction makes
the maximum contribution in temperature dynamics.
The conduction of heat from blood to the skin’s surface involves two components, heat
conduction from blood to biological tissue and the propagation of heat perturbation through
the biological tissue to the skin. Blood flow oscillations are considered to be the source of proportional
temperature perturbations that propagate within skin from a depth of about 1–2 mm
to the skin’s surface; these are called thermal waves (Sagaidachnyi et al 2014). The velocity
of propagation and amplitude attenuation of the thermal wave depend on the medium through
which
---------------------------kilde---------------------------
Thermography-based blood flow imaging in human
skin of the hands and feet: a spectral filtering
approach
-----------------------------------------------------------


photoplethysmography (PPG) which measures infrared radiation absorption of
blood by hemoglobin that circulates in surface tissues

-------------------------------kilde-------
Restoration of finger blood flow oscillations by means of thermal imaging
-----------------------------------------------------------------------


It is well established that skin temperature oscillations in fingertips coexist
with blood flow oscillations and there is a certain correlation between them.
At the same time, the reasons for differences in waveform and the delay
between the blood flow and temperature oscillations are far from being fully
understood.

The blood flow oscillations were
considered as a source of thermal waves propagating from micro-vessels
towards the skin surface and manifesting as temperature oscillations.

The frequency dependences of delay time and amplitude attenuation in
temperature compared with blood flow oscillations have been determined
in endothelial (0.005–0.02 Hz) and neurogenic (0.02–0.05 Hz) frequency
bands using the wavelet spectra.

At rest
the metabolic heat is a relatively steady component, wherein dynamics of skin temperature
depends predominantly on the blood flowperfusion and influence of metabolic heat is the effect
of the second

It is well-known that volumetric blood flowin blood vessels of fingers ismodulated by different
physiological mechanisms of variation in vascular tone (cardiac, respiratory, myogenic,
neurogenic, endothelial and perhaps others), wherein each mechanism impacts within its
frequency interval, thereby the blood flow oscillations have a complex spectral distribution
(Bernjak et al 2008). The volume blood flow oscillations in the surface vessels of the finger
are a source of temperature oscillations originating at some depth (0.5–2 mm from the skin
surface).
---------------------------------kilde -----------------------
Determination of the amplitude and phase
relationships between oscillations in skin
temperature and photoplethysmographymeasured
blood flow in fingertips
---------------------------------------------------------------



Visualisation of the vessels underlying the skin is important for assessment of peripheral vascular
disease, the results of skin reconstructive surgery, wound and ulcer management and
(via perforator vessels) aspects of muscular function. Several non-invasive optical imaging
methods, such as capillaroscopy, laser Doppler, laser speckle contrast and orthogonal polarisation
spectral imaging, have been applied by various groups to quantify functional aspects of
the skin vasculature. These methods are sensitive to vasculature plexuses up to 0.2mm deep
from the skin surface. Because of thermo-conduction of the skin and thermo-convection of
the blood via perforator vessels, infrared (IR) imaging, however, can assess vascular plexuses
at an estimated depth of 10mm below the skin surface [1]. Recently, IR imaging of a subject’s
forearm was utilised to visualise skin angioarchitecture in real time in response to the
infusion of vasodilators [2].
Liu et al. [3] showed that, by using IR imaging during 5 min of complete arterial
occlusion by an inflatable cuff to block blood flow in a human forearm, patterns characteristic
to microvascular (MV) perforator vessels, conduit vessels as well as skin areas without
IR-detectable vessels (SWV) can be distinguished on the skin. However, 5-min long ischemia

brings predictable discomfort and may not always be applicable to subjects with various
health conditions or to pediatric patients. In this study, we attempt to map angioarchitecture
of the skin reliably by evaluating temperature deviations following 1, 4 and 5 min of complete
arterial occlusion of the forearm by an inflatable cuff.

---------------------kilde-------------------------------
Reconstruction of thermographic signals to map perforator vessels in
humans
--------------------------------------------------------- 



Skin blood volume variation can be viewed as a cause of propagating through the tissue so-called heat waves (or
(temperature waves). The heat waves characterized by the frequency-dependant amplitude attenuation and dispersion.
Relationships between spectral components of skin temperature and blood flow oscillations have been established in our
recent study7.


Some fourier transform is used 

dertimination of frequency band 

-------------------------kilde----------------------------
Influence of temporal noise on the skin blood flow measurements
performed by cooled thermal imaging camera:
limit possibilities within each physiological frequency range
-----------------------------------------------------------